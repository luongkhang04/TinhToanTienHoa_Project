\begin{center}
    \textbf{\Large ABSTRACT}
\end{center}
\vspace{1em}
Navigating dynamic environments requires planners that balance long-horizon guidance with real-time collision avoidance under limited sensing. This report presents a two-level hierarchical planner for a holonomic agent in a discrete grid with static and moving obstacles. A global planning layer computes a waypoint route on the static map using BFS, DFS, Dijkstra, or A*, with safety-margin enforcement and waypoint post-processing; a local planning layer continuously replans within a $5\times5$ observation window using reactive search, potential fields, a grid-adapted Dynamic Window Approach (DWA), or a rolling-horizon evolutionary optimizer. The layers are coordinated through waypoint tracking, obstacle inflation, and stuck detection that triggers global replanning when progress stalls. We benchmark 24 global--local combinations on seven maps with 0--200 moving obstacles (672 runs) and measure success rate, path length, smoothness, and runtime. The best combinations reach 96.4\% success, while the overall benchmark success is 61.9\%. Results show that DWA produces the shortest and fastest trajectories, potential fields yield the smoothest paths, and the evolutionary local planner remains highly reliable at the cost of higher computation, highlighting practical trade-offs in hierarchical navigation for dynamic settings.
