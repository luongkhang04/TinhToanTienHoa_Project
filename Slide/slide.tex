% !TEX program = pdflatex
\documentclass[aspectratio=169]{beamer}

% Theme & colors
\usetheme{CambridgeUS}
\usecolortheme{beaver}
\usefonttheme{professionalfonts}

% Encoding & fonts
\usepackage[T1]{fontenc}
\usepackage[utf8]{inputenc}
\usepackage{lmodern}

% Figures/tables
\usepackage{graphicx}
\usepackage{booktabs}
\usepackage{tabularx}

% Math
\usepackage{amsmath,amssymb,mathtools}
\usepackage{bm}

% Paths for figures (slides live in Slide/)
\graphicspath{{images/}{../Report/images/}{../Code/results/}}

% Convenience
\newcommand{\E}{\mathbb{E}}
\newcommand{\Var}{\mathrm{Var}}
\newcommand{\Prb}{\mathbb{P}}

% Metadata
\title[Dynamic path planning]{Path Planning in Dynamic Environments}
\subtitle{A hierarchical global+local planning framework on gridworlds}
\author[Khang Luong, Ong Xuan Son]{Khang Luong \and Ong Xuan Son}
\institute[HUST]{Hanoi University of Science and Technology (HUST)}
\date{\today}

\titlegraphic{\includegraphics[height=1.0cm]{logo-soict-hust-1.png}}

\AtBeginSection[]{
    \begin{frame}{Outline}
        \tableofcontents[currentsection]
    \end{frame}
}

\begin{document}

%------------------------------------
\begin{frame}
    \titlepage
\end{frame}

\begin{frame}{Outline}
    \tableofcontents
\end{frame}

%====================================
\section{Motivation}

\begin{frame}{Why Dynamic Environments Are Hard}
\begin{itemize}
    \item \textbf{Time-varying feasibility:} free space changes over time.
    \item \textbf{Partial observability:} agent only sees a local neighborhood.
    \item \textbf{Real-time constraint:} replanning from scratch can be too slow.
    \item \textbf{Safety:} must avoid both static and moving obstacles online.
\end{itemize}
\vspace{0.3em}
\textbf{Idea:} combine long-horizon structure (global) with short-horizon reaction (local).
\end{frame}

\begin{frame}{Scope \,\&\, Contributions}
\textbf{Setting:} single holonomic agent on a 2D discrete grid with static + moving obstacles.
\vspace{0.6em}

\textbf{Contributions}
\begin{itemize}
    \item Hierarchical planner: \textbf{global waypoint path} + \textbf{local reactive navigation}.
    \item Unified evaluation across global planners (BFS/DFS/Dijkstra/A*) and local planners.
    \item Path post-processing: \textbf{line-of-sight sparsification} + \textbf{safety-margin enforcement}.
\end{itemize}
\end{frame}

%====================================
\section{Problem Formulation}

\begin{frame}{Gridworld Model}
\begin{itemize}
    \item State: $S = \{(x,y)\}$ with discrete space/time.
    \item Actions: $A = \{(0,1),(0,-1),(-1,0),(1,0),(0,0)\}$.
    \item Static obstacles: $O \subset S$.
    \item Moving obstacles vary over time: $M_i$ at time step $i$.
    \item Observation: local $5\times 5$ occupancy grid centered on the agent.
\end{itemize}
\end{frame}

\begin{frame}{Planning Objective}
Find a trajectory $T = \{S_0, S_1, \dots, S_{N-1}\}$ such that:
\begin{itemize}
    \item $S_0$ is start and $S_{N-1}$ reaches the goal.
    \item For each step, $\exists a \in A$ with $f(S_i,a)=S_{i+1}$.
    \item Safety constraints: $S_i \notin O$ and $S_i \notin M_i$.
\end{itemize}
\vspace{0.4em}
\textbf{Challenge:} $M_i$ changes online, so the planner must react during execution.
\end{frame}

%====================================
\section{Methodology}

\begin{frame}{Two-Level Hierarchical Architecture}
\begin{columns}[T,onlytextwidth]
    \column{0.55\textwidth}
    \textbf{Global layer (static map)}
    \begin{itemize}
        \item Computes a waypoint route from start to goal.
        \item Enforces safety margin via clearance map.
        \item Post-process: line-of-sight sparsification, wall-pushing.
    \end{itemize}
    \vspace{0.4em}
    \textbf{Local layer (online, reactive)}
    \begin{itemize}
        \item Plans inside observation window ($r=2 \Rightarrow 5\times 5$).
        \item Avoids moving obstacles, tracks current global waypoint.
        \item Triggers global replanning when stuck.
    \end{itemize}
    \column{0.45\textwidth}
    \includegraphics[width=\linewidth,keepaspectratio]{1.png}
    \vspace{0.2em}
    \scriptsize Example grid map (from benchmark set).
\end{columns}
\end{frame}

\begin{frame}{Global Planners (Waypoints)}
\begin{itemize}
    \item \textbf{Grid BFS}: shortest in number of grid steps (8-connected), with safety margin.
    \item \textbf{Grid DFS}: lower memory; not guaranteed shortest.
    \item \textbf{A*}: $f(n)=g(n)+h(n)$; uses Manhattan heuristic with diagonal moves enabled (not always optimal).
    \item \textbf{Dijkstra}: optimal distances; slower (worst-case $O(|V|^2)$ in this implementation).
\end{itemize}
\vspace{0.3em}
\textbf{Post-processing:} clearance map (multi-source BFS), Bresenham line-of-sight sparsification, wall pushing.
\end{frame}

\begin{frame}{Local Planners (Reactive Navigation)}
\begin{itemize}
    \item \textbf{Reactive BFS}: BFS in local observation graph (fast, robust).
    \item \textbf{Reactive DFS}: DFS locally (can be brittle under dynamics).
    \item \textbf{Potential Field}: attractive + repulsive forces, mapped to 4-connected action.
    \item \textbf{Greedy}: choose locally improving action (myopic; can get stuck).
\end{itemize}
\end{frame}

\begin{frame}{Stuck Detection \,\&\, Replanning}
Execution loop monitors:
\begin{itemize}
    \item No-motion events and no-progress events to current waypoint.
    \item If stuck counter exceeds threshold (typically 15 steps in benchmark):
    \begin{itemize}
        \item Trigger \textbf{global replanning} from current position to final goal.
        \item Reset waypoint index and continue.
    \end{itemize}
\end{itemize}
\vspace{0.3em}
Consistent safety margins: global margin typically 2 cells; local obstacle inflation typically 1 cell.
\end{frame}

%====================================
\section{Experiments \,\&\, Results}

\begin{frame}{Benchmark Setup}
\begin{itemize}
    \item 7 maps, obstacle counts $\{0,50,100,200\}$.
    \item 16 combinations: 4 global \(\times\) 4 local planners $\Rightarrow 448$ runs.
    \item Observation radius $r=2$ ($5\times 5$ window); local inflation margin 1.
    \item Stop: 2000 steps or 100 seconds; success when Manhattan distance to goal $\le 3$.
\end{itemize}
\end{frame}

\begin{frame}{Benchmark Maps}
\centering
\begin{tabular}{cccc}
    \includegraphics[width=0.23\linewidth]{1.png} &
    \includegraphics[width=0.23\linewidth]{2.png} &
    \includegraphics[width=0.23\linewidth]{3.png} &
    \includegraphics[width=0.23\linewidth]{4.png} \\
    \includegraphics[width=0.23\linewidth]{5.png} &
    \includegraphics[width=0.23\linewidth]{6.png} &
    \includegraphics[width=0.23\linewidth]{7.png} &
    \phantom{\includegraphics[width=0.23\linewidth]{7.png}} \\
\end{tabular}
\vspace{0.2em}
\scriptsize White: free space. Black: obstacles.
\end{frame}

\begin{frame}{Overall Results (448 runs)}
\begin{itemize}
    \item Total success: \textbf{237/448} $\Rightarrow$ \textbf{52.9\%}.
    \item Successful runs: average path length \textbf{327.10 $\pm$ 243.60} cells.
    \item Successful runs: average runtime \textbf{5.3429 $\pm$ 5.1084} seconds.
\end{itemize}
\vspace{0.6em}
\centering
\includegraphics[width=0.95\linewidth,height=0.55\textheight,keepaspectratio]{success_rates.png}
\end{frame}

\begin{frame}{Key Finding: Local Method Dominates Reliability}
\begin{itemize}
    \item \textbf{Potential field} combinations rank highest in success rate.
    \item \textbf{Reactive BFS} is competitive but consistently below potential field.
    \item \textbf{Greedy} gives mid-range success.
    \item \textbf{Reactive DFS fails across all globals} (0/28 per combination).
\end{itemize}
\end{frame}

\begin{frame}{Path Quality \,\&\, Runtime (Successful Runs)}
\centering
\includegraphics[width=0.95\linewidth,height=0.78\textheight,keepaspectratio]{metrics_comparison.png}
\end{frame}

\begin{frame}{Impact of Obstacle Density}
\centering
\includegraphics[width=0.95\linewidth,height=0.78\textheight,keepaspectratio]{obstacle_impact.png}
\end{frame}

\begin{frame}{Heatmap Summary (Averages)}
\centering
\includegraphics[width=0.95\linewidth,height=0.78\textheight,keepaspectratio]{algorithm_heatmap.png}
\end{frame}

%====================================
\section{Conclusion}

\begin{frame}{Conclusions}
\begin{itemize}
    \item A hierarchical \textbf{global waypoint + local reactive} planner works well under moving obstacles.
    \item In this benchmark, \textbf{local planning choice} is the biggest factor for success.
    \item Potential-field local planning achieves the best overall robustness and trajectory smoothness.
\end{itemize}
\vspace{0.6em}
\textbf{Future work}
\begin{itemize}
    \item Improve local minima handling for potential fields.
    \item Use admissible heuristic for diagonal A* (e.g., octile distance).
    \item Stronger dynamic-obstacle prediction and richer action set.
\end{itemize}
\end{frame}

\begin{frame}{Q\,\&\,A}
\centering
\Large Questions?
\end{frame}

\end{document}