% !TEX program = pdflatex
\documentclass[aspectratio=169]{beamer}

% Theme & colors
\usetheme{CambridgeUS}
\usecolortheme{beaver}
\usefonttheme{professionalfonts}

% Encoding & fonts
\usepackage[T1]{fontenc}
\usepackage[utf8]{inputenc}
\usepackage{lmodern}

% Figures/tables
\usepackage{graphicx}
\usepackage{booktabs}
\usepackage{tabularx}

% Math
\usepackage{amsmath,amssymb,mathtools}
\usepackage{bm}

% Paths for figures (slides live in Slide/)
\graphicspath{{images/}{../Report/images/}{../Code/results/}}

% Convenience
\newcommand{\E}{\mathbb{E}}
\newcommand{\Var}{\mathrm{Var}}
\newcommand{\Prb}{\mathbb{P}}

% Metadata
\title[Dynamic path planning]{Path Planning in Dynamic Environments}
\subtitle{A hierarchical global+local planning framework on gridworlds}
\author[Khang Luong, Ong Xuan Son]{Khang Luong \and Ong Xuan Son}
\institute[HUST]{Hanoi University of Science and Technology (HUST)}
\date{\today}

\titlegraphic{\includegraphics[height=1.0cm]{logo-soict-hust-1.png}}

\AtBeginSection[]{
    \begin{frame}{Outline}
        \tableofcontents[currentsection]
    \end{frame}
}

\begin{document}

%------------------------------------
\begin{frame}
    \titlepage
\end{frame}

\begin{frame}{Outline}
    \tableofcontents
\end{frame}

%====================================
\section{Motivation}

\begin{frame}{Why Dynamic Environments Are Hard}
\begin{itemize}
    \item \textbf{Time-varying feasibility:} free space changes over time.
    \item \textbf{Partial observability:} agent only sees a local neighborhood.
    \item \textbf{Real-time constraint:} replanning from scratch can be too slow.
    \item \textbf{Safety:} must avoid both static and moving obstacles online.
\end{itemize}
\vspace{0.3em}
\textbf{Idea:} combine long-horizon structure (global) with short-horizon reaction (local).
\end{frame}

\begin{frame}{Scope \,\&\, Contributions}
\textbf{Setting:} single holonomic agent on a 2D discrete grid with static + moving obstacles.
\vspace{0.6em}

\textbf{Contributions}
\begin{itemize}
    \item Hierarchical planner: \textbf{global waypoint path} + \textbf{local reactive navigation}.
    \item Unified evaluation across 24 combinations of 4 global planners and 6 local planners.
\end{itemize}
\end{frame}

%====================================
\section{Problem Formulation}

\begin{frame}{Gridworld Model}
\begin{align*}
	\text{State space:}\quad & S = \{(x,y) \mid x,y \in \mathbb{N},\ 0 \leq x \leq X_{\text{dim}},\ 0 \leq y \leq Y_{\text{dim}}\} \subset \mathbb{N}^2, \\
	\text{Action space:}\quad & A = \{(0,1),\ (0,-1),\ (-1,0),\ (1,0),\ (0,0)\} \subset \mathbb{Z}^2, \\
	\text{Observation space:}\quad & G \in \{0,1\}^{5 \times 5}, \\
	\text{Static obstacles:}\quad & O \subset S.\\
    \text{Dynamic obstacles:}\quad & M_t \subset S \text{ at time } t.\\
    \text{Transition function:}\quad & f : S \times A \rightarrow S.
\end{align*}
\end{frame}

\begin{frame}{Planning Objective}
Find a trajectory $T = \{S_0, S_1, \dots, S_{N-1}\}$ such that:
\begin{itemize}
    \item $S_0$ is start and $S_{N-1}$ reaches the goal.
    \item For each step, $\exists a \in A$ with $f(S_i,a)=S_{i+1}$.
    \item Safety constraints: $S_i \notin O$ and $S_i \notin M_i$.
\end{itemize}
\vspace{0.4em}
\textbf{Challenge:} $M_i$ changes online, so the planner must react during execution.
\end{frame}

%====================================
\section{Methodology}

\begin{frame}{Two-Level Hierarchical Architecture}
\begin{columns}[T,onlytextwidth]
    \column{0.55\textwidth}
    \textbf{Global layer (static map)}
    \begin{itemize}
        \item Computes a waypoint route from start to goal.
        \item Enforces safety margin via clearance map.
        \item Post-process: line-of-sight sparsification, wall-pushing.
    \end{itemize}
    \vspace{0.4em}
    \textbf{Local layer (online, reactive)}
    \begin{itemize}
        \item Plans inside observation window ($r=2 \Rightarrow 5\times 5$).
        \item Avoids moving obstacles, tracks current global waypoint.
        \item Triggers global replanning when stuck.
    \end{itemize}
    \column{0.45\textwidth}
    \includegraphics[width=\linewidth,keepaspectratio]{planning.png}
    \vspace{0.2em}
    \scriptsize Example grid map (from benchmark set).
\end{columns}
\end{frame}

\begin{frame}{Global Planners (Waypoints)}
\begin{itemize}
    \item \textbf{Grid BFS}: shortest in number of grid steps (8-connected), with safety margin.
    \item \textbf{Grid DFS}: lower memory; not guaranteed shortest.
    \item \textbf{A*}: $f(n)=g(n)+h(n)$; uses Manhattan heuristic with diagonal moves enabled.
    \item \textbf{Dijkstra}: optimal distances; slower.
\end{itemize}
\vspace{0.3em}
\textbf{Post-processing:} clearance map, Bresenham line-of-sight sparsification, wall pushing.
\end{frame}

\begin{frame}{Local Planners (Reactive Navigation)}
\begin{itemize}
    \item \textbf{Reactive BFS}: BFS in local observation graph.
    \item \textbf{Reactive DFS}: DFS locally.
    \item \textbf{Potential Field}: attractive + repulsive forces.
    \item \textbf{Greedy}: choose locally improving action.
    \item \textbf{DWA}: simulate short trajectories, pick best.
    \item \textbf{Evolutionary}: optimizes a sequence of actions \(\pi=(a_0,\dots,a_{H-1})\) over a 5-action set.
\end{itemize}
\end{frame}

\begin{frame}{Local Planner: Potential Field}
    \begin{columns}[T,onlytextwidth]
    \column{0.6\textwidth}
    \begin{itemize}
        \item Computes an attractive force toward the projected waypoint and repulsive forces away from obstacles.
        \item Net force: \(\bm{F}=\bm{F}_{att}+\bm{F}_{rep}\).
        \item Maps \(\bm{F}\) to a \textbf{4-connected} action (dominant axis) while respecting the margin constraint.
    \end{itemize}
    \column{0.4\textwidth}
    \includegraphics[width=\linewidth,keepaspectratio]{potential field.jpg}
\end{columns}
\end{frame}

\begin{frame}{Local Planner: Dynamic Window Approach (DWA)}
\begin{itemize}
    \item Samples discrete velocity commands within a \textbf{window} around the previous action
    \begin{equation}
    v_x \in [v_{x,t-1}-a_{max},\, v_{x,t-1}+a_{max}], \quad
    v_y \in [v_{y,t-1}-a_{max},\, v_{y,t-1}+a_{max}],
    \end{equation}
    \item Simulates short trajectories (prediction horizon).
    \begin{equation}
    \vec{p}_k = \vec{p}_0 + k\vec{v}, \quad k=1,\dots,N
    \end{equation}
    \item Scores candidates by goal heading, obstacle clearance, and speed.
    \begin{equation}
    J(\vec{v}) = w_h \, S_{heading} + w_c \, S_{clearance} + w_s \, S_{speed}
    \end{equation}
    \[S_{heading} = \frac{1}{\|\vec{p}_{final}-\vec{p}_{goal}\|+1}
    \quad S_{clearance} = \frac{\min(c_{min}, c_{cap})}{c_{cap}}
    \quad S_{speed} = \frac{k\|\vec{v}\|}{v_{max}}\]
\end{itemize}
\end{frame}

\begin{frame}{Local Planner: Rolling-Horizon Evolutionary}
\begin{itemize}
    \item Optimizes a short sequence of actions \(\pi=(a_0,\dots,a_{H-1})\) over a 5-action set.
    \item Fitness combines distance-to-goal terms, clearance risk, turn/stall penalties, and unknown-space penalty.
    \[
    d_t = |x_t - x_g| + |y_t - y_g|,
    \quad d_{final} = d_H, 
    \quad d_{min} = \min_{1 \le t \le H} d_t, 
    \quad \bar{d} = \frac{1}{H}\sum_{t=1}^{H} d_t
    \]
    \[
    T = \sum_{t=1}^{H-1} \mathbb{I}[a_{t-1} \neq \texttt{NONE} \wedge a_t \neq \texttt{NONE} \wedge a_{t-1} \neq a_t], 
    \quad S = \sum_{t=0}^{H-1} \mathbb{I}[a_t = \texttt{NONE}]
    \]
    \item Evolves a population via elitism, tournament selection, crossover, and mutation.
\end{itemize}
\end{frame}

%====================================
\section{Experiments \,\&\, Results}

\begin{frame}{Benchmark Setup}
\begin{itemize}
    \item 7 maps, obstacle counts $\{0,50,100,200\}$.
    \item 24 combinations: 4 global \(\times\) 6 local planners $\Rightarrow 672$ runs.
    \item Observation radius $r=2$ ($5\times 5$ window); local inflation margin 1.
    \item Stop: 2000 steps or 100 seconds; success when Manhattan distance to goal $\le 3$.
\end{itemize}
\end{frame}

\begin{frame}{Benchmark Maps}
\centering
\begin{tabular}{cccc}
    \includegraphics[width=0.23\linewidth]{1.png} &
    \includegraphics[width=0.23\linewidth]{2.png} &
    \includegraphics[width=0.23\linewidth]{3.png} &
    \includegraphics[width=0.23\linewidth]{4.png} \\
    \includegraphics[width=0.23\linewidth]{5.png} &
    \includegraphics[width=0.23\linewidth]{6.png} &
    \includegraphics[width=0.23\linewidth]{7.png} &
    \phantom{\includegraphics[width=0.23\linewidth]{7.png}} \\
\end{tabular}
\vspace{0.2em}
\scriptsize White: free space. Black: obstacles.
\end{frame}

\begin{frame}{Overall Results (672 runs)}
\centering
\includegraphics[width=\linewidth,height=0.85\textheight,keepaspectratio]{success_rates.png}
\end{frame}

\begin{frame}{Path Quality \,\&\, Runtime (Successful Runs)}
\centering
\includegraphics[width=\linewidth,height=0.85\textheight,keepaspectratio]{metrics_comparison.png}
\end{frame}

\begin{frame}{Impact of Obstacle Density}
\centering
\includegraphics[width=\linewidth,height=0.85\textheight,keepaspectratio]{obstacle_impact.png}
\end{frame}

\begin{frame}{Heatmap Summary (Averages)}
\centering
\includegraphics[width=\linewidth,height=0.85\textheight,keepaspectratio]{algorithm_heatmap.png}
\end{frame}

%====================================

\begin{frame}{Q\,\&\,A}
\centering
\Large Questions?
\end{frame}

\end{document}
